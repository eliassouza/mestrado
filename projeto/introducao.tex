\section{Apresentação da problemática}

O Brasil apresenta altos índices de crescimento agrícola e com produção que vem crescendo constantemente. O sítio governamental \textbf{Portal Brasil}\footnote{http://www.brasil.gov.br/ - Acesso em 10. set. 2013} traz dados sobre a relevância do agronegócio para a economia do país e ressalta que o setor é responsável por mais de 22\% do Produto Interno Bruto brasileiro, assim como apresenta números expressivos em relação às vendas para o exterior, principalmente nas relações com a União Européia, China e Estados Unidos. Este Portal reforça que a política agrícola brasileira incentiva a expansão do setor, por meio da concessão de crédito e benefícios fiscais, além de programas como o "seguro rural".

O estado do Maranhão tem participação ativa e crescente no cenário do agronegócio brasileiro. As novas projeções para o agronegócio do Ministério da Agricultura Pecuária e Abastecimento \citeauthor{brministerioAgricultura2013} (\citeyear{brministerioAgricultura2013}) indicam parte da região de abrangência do bioma cerrado nos estados do Maranhão, Piauí, Tocantins e Bahia, como uma das mais promissoras para o agronegócio do pais. Essa região se destaca atualmente principalmente pela produção de soja e arroz e tem uma dinâmica diferenciada de crescimento.

\begin{citacao}
"Seu crescimento tem sido extraordinário. A última pesquisa do IBGE (2011) sobre o PIB municipal mostra que esses 
municípios têm puxado o crescimento dos estados onde se localizam. Seu crescimento tem sido muito maior do que o crescimento do estado e da média brasileira. Esses quatro estados devem atingir uma produção de grãos de 18 milhões de toneladas nos próximos 10 anos numa área plantada de 7,3 milhões de hectares em 2022/2023, mas que poderá atingir 10,5 milhões de hectares em seu limite superior ao final da próxima década. [...] As áreas que vem sendo ocupadas nesses estados têm algumas características essenciais para a agricultura moderna. São planas e extensas, solos potencialmente produtivos, disponibilidade de água, e clima propício com dias longos e com elevada intensidade de sol. A limitação maior, no entanto são as precárias condições de logística, especialmente transporte terrestre, portuário, comunicação e, em algumas áreas ausência de serviços financeiros."
\cite[p. 64]{brministerioAgricultura2013}
\end{citacao}

As projeções ainda indicam que essa área deverá apresentar (\citeauthor{brministerioAgricultura2013} \citeyear{brministerioAgricultura2013}, p. 71) "aumento elevado da produção de grãos assim como sua área deve apresentar também aumento expressivo. [...] Essa região deverá produzir próximo de 18 milhões de toneladas de grãos em 2023 (aumento de 21,6\%) e uma área plantada de grãos entre 7 e 10 milhões de hectares ao final do período das projeções".

A partir desse panorama para o agronegócio na região ocupada pelo cerrado, concentraremos nossos estudos na porção sul do Maranhão, que está contida na área supracitada. Essa escolha se justifica dado que o agronegócio é um setor bastante estimulado da economia maranhense, à medida que gera importantes recursos financeiros para o estado e é encarado pelo poder público como uma espécie de “salvação” para a região, historicamente marcada pela pobreza e grande concentração de renda. 

Todavia, como lembram \citeauthoronline{rodrigues_alencar},

\begin{citacao}
a expansão do agronegócio da soja, a partir de grandes fazendas vem sendo responsável por vários problemas sócio-ambientais que se intensificam no Maranhão. A pressão que a abertura de grandes fazendas está fazendo nos recursos naturais do Cerrado, bem como a expropriação dos pequenos produtores e campesinos da região gera ainda mais problemas nas áreas de expansão da soja. O conflito entre fazendeiros e o campesinato, que tem feito sua história nesses lugares, muitas vezes há séculos, pode ser descrito como a face mais evidente da tensão, e que pode traduzir as várias formas do capitalismo entrar em conflito com as populações tradicionais \cite[p. 1]{rodrigues_alencar}
\end{citacao}

Com princípios de produtividade estabelecidos na “Revolução Verde”, o campo brasileiro vem se modernizando desde a década de 1970 e atingindo formas peculiares de produção. As novas técnicas de produção (maquinário, estradas, ferrovias, portos, pesquisas) artificializam o espaço, criando o meio técnico-científico-informacional de \citeauthoronline{santos1996} (\citeyear{santos1996}), onde os processos espaciais que produzem modificação da natureza são intensificados e criam formas atípicas e exógenas aos lugares:

\begin{citacao}
As modificações que acontecem no espaço agrário maranhense na atualidade são frutos da correlação de forças entre grandes projetos instituídos pelo planejamento estatal como modernizador do espaço maranhense e as populações tradicionais. No Cerrado essa hegemonia pelo território é dada pela tensão entre grandes proprietários e camponeses. A soja adentra esse quadro de modificações e arranjos espaciais como produto da 
modernidade, e em face da necessidade do modo de produção capitalista, em meio às transformações ocorridas nas últimas décadas, como parte de um processo de internacionalização dos espaços nacionais  \cite[p. 3]{rodrigues_alencar}
\end{citacao}

Como lembra \citeauthoronline{studte2008} (\citeyear{studte2008}), desde a década de 1980, a região sul do Maranhão foi transformando suas estruturas tradicionais de agricultura de subsistência em agricultura industrializada, que invariavelmente ocupa grandes extensões do territórios, substituindo ou pressionando áreas tradicionalmente destinadas a produção familiar camponesa. Esta transformação acontece em toda a área de fronteira agrícola do sul da Amazônia, de forma cada vez mais profunda dentro dos biomas amazônico e do cerrado. Ainda segundo o autor (\citeyear{studte2008}, p.15), nessa região, a agricultura mecanizada só é apropriada nas altas serras, que são interrompidas por grandes vales que dificultam a entrada de máquinas.  São nesses vales (também chamados de baixões) que a agricultura de subsistência é praticada. Fora dos vales, a região é realmente atraente para o plantio de monoculturas como a soja, porque as áreas extensas são predominantemente planas. 

\citeauthoronline{alves2006} (\citeyear{alves2006}), ao analisar o agronegócio no cerrado piauiense, chama atenção para a mesma situação, com destaque para os impactos na vida do agricultor tradicional:

\begin{citacao}
tal avanço dos cultivos da agricultura moderna, que por enquanto ocorre nos platôs planos, apresenta reflexos ambientais negativos também sobre os baixões, o que acarreta problemas na mesma intensidade para os moderadores dessas áreas. Nos últimos anos, tal população vem observando uma diminuição da vazão de água dos riachos por conta da eliminação das nascentes nos platôs, problema que se avoluma com o assoreamento dos cursos d'água decorrentes da erosão produzida, na área de cultivo, com o desmatamento. O impacto mais grave, no entanto, que sofre a população dos baixões é com a contaminação da água e do ar pelos agrotóxicos pulverizados por aviões e máquinas agrícolas sobre as lavouras modernas, mas que atingem os vales, envenenando pessoas e animais em amplas áreas da região. Constata-se, ainda, que a intensificação do uso de agrotóxicos reverteu-se em proliferação de determinadas pragas destruidoras das pequenas lavouras dos camponeses. \cite[p. 278-279]{alves2006}
\end{citacao}

Essa transformação do uso do solo também ocorre em São Raimundo das Mangabeiras, no Maranhão, município objeto de nossa pesquisa. Segundo \citeauthoronline{lima_locatel_silva} (\citeyear{lima_locatel_silva}, p.9-10), ao utilizar dados do Censo Agropecuário do IBGE (2006)\footnote{Disponível em:
http://www.sidra.ibge.gov.br/bda/tabela/listabl.asp?c=788\&z=t\&o=3. Acesso em: 30. jun. 2012.} e analisar a estrutura fundiária deste município, constata-se que na área total do município ocupada por estabelecimentos agropecuários (119.268 ha) existem 764 estabelecimentos. Destes, a maior parte apresenta uma área inferior a 100 hectares (536 estabelecimentos), considerados como unidades de produção familiar, e 228 propriedades ocupam 100.424 hectares, cerca de 84\% do total destinado à estabelecimentos agropecuários no município. 

Esses números indicam que diferentes segmentos da sociedade têm acesso à terra de forma desigual, com as grandes áreas nas mãos de poucos produtores. Os autores ainda ressaltam que

\begin{citacao}
essa reestruturação territorial – expansão do agronegócio, em detrimento do campesinato – ocasionou a precarização do trabalho e das condições de vida da população camponesa, bem como a reprodução da pobreza no campo e na cidade. [...] Com suas terras expropriadas, os camponeses passaram a constituir o proletariado urbano, ocupando as periferias de municípios vizinhos, de modo que a pobreza rural não se atenuou, apenas foi relocalizada. A reprodução da pobreza persiste porque apesar desses investimentos de capital e geração de empregos no sul maranhense representarem uma possibilidade de desenvolvimento social, o que na realidade ocorre é concentração da riqueza gerada pela atividade.
\cite[p. 12-16]{lima_locatel_silva}
\end{citacao}

\citeauthoronline{studte2008} corrobora com esta análise, ao relacionar a migração para cidade com o desemprego. Segundo o autor (\citeyear{studte2008}, p. 11), "uma produção avançada com alta demanda de capital, resulta em uma não demanda de trabalhadores. Se não há absorção no mercado de trabalho, estas pessoas vão migrar para as cidades resultando em problemas sociais, pois as regiões são pouco preparadas para este tipo de migração".

Já \citeauthoronline{lima_locatel_silva} (\citeyear{lima_locatel_silva}, p.15) afirmam que as principais características que marcam o campo brasileiro na atualidade e, que são verificadas na região sul do Maranhão, correspondem à pauperização dos trabalhadores rurais e da estrutura fundiária concentrada e desigual, problemas estes que não foram resolvidos, mas "escamoteados a partir do desenvolvimento do capitalismo no campo".

Para \citeauthoronline{almeida} (\citeyear{almeida}), a reconceituação de território atual tem sido marcada por novos critérios de classificação que reeditam a prevalência do quadro natural, "privilegiando biomas e ecossistemas como delimitadores de "regiões", flexibilizando as normas jurídicas que asseguram os direitos territoriais de povos e comunidades tradicionais e objetivando atender às demandas progressivas de um crescimento econômico baseado principalmente em commodities minerais e agrícolas” \cite{almeida}.

\citeauthoronline{rodrigues_alencar} também relatam que essa nova realidade de produção do espaço maranhense causa conflitos entre os camponeses e os agentes do agronegócio. Segundo os autores, 

\begin{citacao}
a luta do camponês contra o avanço do agronegócio se dá para ele como forma de se manter com seus meios de subsistência, para assim poder ser o protagonista da sua própria história, e não subordinar sua vida as demandas do grande capital, exógeno ao seu lugar e a sua cultura. [...] o conflito se manifesta de várias formas, primeiramente o conflito espacial pela produção, que se dá na forma velada, como anteriormente exposto, mas a forma violenta é a mais cruel, pois desaloja, desacredita e deixa órfãos. \cite[p. 10]{rodrigues_alencar}
\end{citacao}

\citeauthoronline{almeida} (\citeyear{almeida}), do mesmo modo, analisa as ações políticas que originam esse conflito entre produtor e camponês, afirmando que as pressões políticas que articulam a ação governamental visam uma "organização hierarquizada dos territórios". São ações rápidas com objetivos de curto prazo que exigem prontos "resultados estruturais (hidrelétricas, gasodutos, minerodutos, hidrovias, rodovias, portos, aeroportos, linhas de transmissão de energia), cujos efeitos referem-se a acidulados debates jurídicos e à intensificação de conflitos sociais" \cite{almeida}

Para o autor, o ritmo da ação governamental, aliada aos interesses privados que promovem a expansão das commodities, dá fundamento para pressões políticas em todo o país, que se manifestam através do mercado de terras e privilegiam algumas formas de ação, tais como: a) a privatização das terras públicas sob o título de regularização fundiária; b) a redução de áreas protegidas ou unidades de conservação; c) tentativas de incorporação de novas extensões territoriais aos circuitos mercantis - reforma do código florestal e redução das faixas de fronteira; d) a flexibilização dos direitos territoriais de povos e comunidades tradicionais.

Porém, o projeto de ocupação econômica dos cerrados maranhenses pode ser caracterizado

\begin{citacao}
pela negação das populações que aí se encontram, com a negação de sua cultura, identidade, e produção. Nos discursos dos programas de financiamento da agricultura da soja os espaços que estes se expandem são tidos como “Áreas de Cerrado Incorporadas ao Processo Produtivo", implicando uma clara concepção de "espaços vazios” [...] O Cerrado acaba sendo devastado pela paisagem homogênea e tecnificada que é criada,  também tem a diversidade social e cultural dos "Povos do Cerrado" comprometida. \cite[p. 13]{rodrigues_alencar}
\end{citacao}

\citeauthoronline{meiners2013} (\citeyear{meiners2013}) destaca que o intenso desmatamento na região sul maranhense tem como principal causa o agronegócio capitalista de monoculturas. Segundo o autor (\citeyear{meiners2013}, p. 159), essas regiões "não têm projetos estruturados e integrais de conservação, são considerados como um vazio demográfico, o que preocupa o futuro do agroextrativismo nestas regiões, de que formas se adaptarão às pressões do capitalismo e que estratégias de camponização e re-camponização acontecerão".

Conforme \citeauthoronline{studte2008} (\citeyear{studte2008}), quase toda a área de vegetação natural restante na região encontra-se em áreas de ocupação dos pequenos produtores. Tal autor ressalta que é possível verificar que a modernização  da  agricultura na região do sul do Maranhão, sobretudo, sob o comando das grandes empresas foi concretizada  com foco no rendimento, na produtividade, no lucro e na economia de mão-de-obra de trabalhadores rurais.  Ao passo que, os agricultores familiares não conseguem acompanhar esta velocidade de modernização.

Com base no exposto, propomos trabalhar com o Sindicato dos Trabalhadores Rurais e com a Cooperativa Agroecológica pela Vida (COPEVIDA) \footnote{http://cirandas.net/copevida}, representantes da sociedade civil organizada do município São Raimundo das Mangabeiras (MA). Buscamos com a participação desses sujeitos identificar e mapear áreas que estão sendo alteradas pela ação do agronegócio, as quais são importantes para a reprodução da vida das comunidades camponesas afetadas. Dessa forma, visamos o apoderamento da técnica por parte da sociedade civil, buscando reduzir as diferenças de poder de acesso à informação em relação às grandes corporações.

Em tempos de desenvolvimento do agronegócio e conflitos de terra, o detentor da técnica exerce maior poder, legitimando seus interesses no território. Se a sociedade está, ainda que de forma seletiva, cada vez mais conectada à rede, acreditamos ser de fundamental importância a possibilidade de conexão dos movimentos sociais da região à rede, para que possam trabalhar com a informação com um potencial, mesmo que em menor escala, parecido com o dos agentes hegemônicos. Dessa forma, propomos um projeto com intenção fortalecer a inclusão dos movimentos sociais nas redes informacionais, permitindo o acesso à técnica pelos menos favorecidos.

