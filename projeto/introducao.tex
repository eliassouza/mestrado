\chapter{Introdução e justificativa}

O Brasil apresenta altos índices de desenvolvimento agrícola e com produção que vem crescendo constantemente. O sítio governamental \textbf{Portal Brasil}\footnote{http://www.brasil.gov.br/ - Acesso em 10. set. 2013} traz dados sobre a relevância do agronegócio para a economia do país e ressalta que o setor é responsável por mais de 22\% do Produto Interno Bruto brasileiro, assim como apresenta números expressivos em relação às vendas para o exterior, principalmente nas relações com a União Européia, China e Estados Unidos. O \textbf{Portal Brasil} reforça que a política agrícola brasileira incentiva a expansão do setor, por meio da concessão de crédito e benefícios fiscais, além de programas como o "seguro rural".

O estado do Maranhão tem participação ativa e crescente no cenário do agronegócio brasileiro. O periódico “O Povo”, em seu caderno Economia do Nordeste\footnote{http://economianordeste.opovo.com.br/estados/ma/ - Acesso em 30. ago. 2013}, frisa que em 2007, o Maranhão teve uma participação de 0,50\% nas exportações do agronegócio brasileiro, ao fechar o ano com um volume de US\$ 290,5 milhões. Em 2008, esse valor quase dobra, graças a produção de soja (que registrou saldo de US\$ 447 milhões distribuídos entre grãos, farelo e óleo). Em segundo lugar aparece a madeira, com um volume de US\$ 9 milhões. 
O agronegócio e a pecuária iniciaram um processo de expansão, focados no aumento da produtividade e na qualidade dos rebanhos. Arroz, soja, milho, algodão, cana-de-açúcar e biomassa atraem cada vez mais investidores.

Como vimos, o agronegócio é um setor da economia maranhense bastante estimulado, à medida que gera importantes recursos financeiros para o estado e é encarado pelo poder público como uma espécie de “salvação” para a região, historicamente marcada pela pobreza e concentração de renda. Porém, como lembram \citeauthoronline{rodrigues_alencar},

\begin{citacao}
a expansão do agronegócio da soja, a partir de grandes fazendas vem sendo responsável por vários problemas sócio-ambientais que se intensificam no Maranhão. A pressão que a abertura de grandes fazendas está fazendo nos recursos naturais do Cerrado, bem como a expropriação dos pequenos produtores e campesinos da região gera ainda mais problemas nas áreas de expansão da soja. O conflito entre fazendeiros e o campesinato, que tem feito sua história nesses lugares, muitas vezes há séculos, pode ser descrito como a face mais evidente da tensão, e que pode traduzir as várias formas do capitalismo entrar em conflito com as populações tradicionais \cite[p. 1]{rodrigues_alencar}
\end{citacao}

Com princípios de produtividade estabelecidos na “Revolução Verde”, o campo brasileiro vem se modernizando desde a década de 1970 e atingindo formas peculiares de produção. As novas técnicas de produção (maquinário, estradas, ferrovias, portos, pesquisas) artificializam o espaço, criando o meio técnico-científico-informacional de \citeauthoronline{santos1996} (\citeyear{santos1996}), onde os processos espaciais de modificação da natureza são intensificados e criam formas atípicas e exógenas aos lugares:

\begin{citacao}
As modificações que acontecem no espaço agrário maranhense na atualidade são frutos da correlação de forças entre grandes projetos instituídos pelo planejamento estatal como modernizador do espaço maranhense e as populações tradicionais. No Cerrado essa hegemonia pelo território é dada pela tensão entre grandes proprietários e camponeses. A soja adentra esse quadro de modificações e arranjos espaciais como produto da 
modernidade, e em face da necessidade do modo de produção capitalista, em meio às transformações ocorridas nas últimas décadas, como parte de um processo de internacionalização dos espaços nacionais  \cite[p. 3]{rodrigues_alencar}
\end{citacao}

Como lembra \citeauthoronline{studte2008} (\citeyear{studte2008}), desde a década de 1980, a região sul do Maranhão foi transformando suas estruturas tradicionais de agricultura de subsistência em agricultura industrializada, que invariavelmente ocupa territórios maiores do que o modo de subsistência. Esta transformação acontece em toda fronteira sul da Amazônia, de forma cada vez mais profunda dentro dos biomas amazônico e do cerrado.

Para Alfredo Wagner de \citeauthoronline{almeida}, a reconceituação de território atual tem sido marcada por novos critérios de classificação que reeditam a prevalência do quadro natural, “privilegiando biomas e ecossistemas como delimitadores de "regiões", flexibilizando as normas jurídicas que asseguram os direitos territoriais de povos e comunidades tradicionais e objetivando atender às demandas progressivas de um crescimento econômico baseado principalmente em commodities minerais e agrícolas” \cite{almeida}.

\citeauthoronline{rodrigues_alencar} também relatam que essa nova realidade de produção do espaço maranhense causa conflitos entre os camponeses e os agentes do agronegócio. Segundo os autores, 

\begin{citacao}
a luta do camponês contra o avanço do agronegócio se dá para ele como forma de se manter com seus meios de subsistência, para assim poder ser o protagonista da sua própria história, e não subordinar sua vida as demandas do grande capital, exógeno ao seu lugar e a sua cultura. [...] o conflito se manifesta de várias formas, primeiramente o conflito espacial pela produção, que se dá na forma velada, como anteriormente exposto, mas a forma violenta é a mais cruel, pois desaloja, desacredita e deixa órfãos. \cite[p. 10]{rodrigues_alencar}
\end{citacao}

\citeauthoronline{almeida} (\citeyear{almeida}), analisa as ações políticas que originam esse conflito entre produtor e camponês, afirmando que as pressões políticas que articulam a ação governamental visam uma "organização hierarquizada dos territórios". São ações rápidas com objetivos de curto prazo que exigem prontos "resultados estruturais (hidrelétricas, gasodutos, minerodutos, hidrovias, rodovias, portos, aeroportos, linhas de transmissão de energia), cujos efeitos referem-se a acidulados debates jurídicos e à intensificação de conflitos sociais" \cite{almeida}

Para o autor, o ritmo da ação governamental, aliada aos interesses privados que promovem a expansão das commodities, dá fundamento para pressões políticas em todo o país, que se manifestam através do mercado de terras e privilegiam algumas formas de ação:

\begin{alineas}
   \item a privatização das terras públicas sob o título de regularização fundiária;
   \item a redução de áreas protegidas ou unidades de conservação;
   \item a tentativas de incorporação de novas extensões territoriais aos circuitos mercantis - reforma do código florestal e redução das faixas de fronteira;
   \item a flexibilização dos direitos territoriais de povos e comunidades tradicionais.
\end{alineas}

Porém, o projeto de ocupação econômica dos cerrados maranhenses pode ser caracterizado

\begin{citacao}
pela negação das populações que aí se encontram, com a negação de sua cultura, identidade, e produção. Nos discursos dos programas de financiamento da agricultura da soja os espaços que estes se expandem são tidos como “Áreas de Cerrado Incorporadas ao Processo Produtivo", implicando uma clara concepção de "espaços vazios” [...] O Cerrado acaba sendo devastado pela paisagem homogênea e tecnificada que é criada,  também tem a diversidade social e cultural dos "Povos do Cerrado" comprometida. \cite[p. 13]{rodrigues_alencar}
\end{citacao}

Conforme \citeauthoronline{studte2008} (\citeyear{studte2008}), quase toda a área de vegetação natural restante na região está na propriedade dos pequenos produtores. O autor ressalta que  é possível verificar que a modernização da agricultura na região sul do Maranhão na direção das grandes empresas foi concretizada com foco no rendimento, produtividade, no lucro e na economia de mão-de-obra de trabalhadores rurais. Os agricultores tradicionais, de pequeno porte, não conseguem acompanhar esta velocidade de modernização. 

O autor analisa também a relação entre o índice de desenvolvimento humano (IDH) e o agronegócio na região

\begin{citacao}
Entre os municípios com o maior IDH estão os grandes produtores de soja como Balsas, São Raimundo das Mangabeiras e Fortaleza das Nogueiras. Balsas é o município com a maior área plantada de soja no Maranhão e um dos municípios com o maior IDH de todos. Por enquanto isso significa, pelo menos, que a produção de soja não tem efeito negativo para um índice que abrange saúde, educação e renda. \cite[p. 13]{studte2008}
\end{citacao} 


Em tempos de desenvolvimento do agronegócio e conflitos de terra, o detentor da técnica exerce maior poder, legitimando seus interesses no território. Se a sociedade está, ainda que de forma seletiva, cada vez mais conectada à rede, acreditamos ser de fundamental importância a possibilidade de conexão dos movimentos sociais da região à rede, para que possam trabalhar com a informação com um potencial parecido com o dos agentes hegemônicos. Dessa forma, propomos um projeto com intenção fortalecer a inclusão dos movimentos sociais nas redes informacionais, permitindo o acesso à técnica pelos menos favorecidos.
