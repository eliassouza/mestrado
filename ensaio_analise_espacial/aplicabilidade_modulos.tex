\section{Aplicabilidade do conteúdo dos módulos da disciplina na pesquisa proposta}

\subsection{Módulo 1 - Conceitos e paradigmas da análise geoespacial na geografia}

Conforme relatei no capítulo anterior, acredito que as lições sobre planejamento de pesquisa seja o grande diferencial desse módulo. Como trabalho com a expansão do agronegócio e como a sociedade civil está se organizando para combater essa expansão, acredito que teria que planejar minha pesquisa em torno de alguns questionamentos básicos:

\begin{enumerate}
  \item Como era a ocupação do solo na região de estudo antes da Revolução Verde no Brasil (anos 1970)?
  \item Como é a ocupação do solo na região de estudo no período atual?
  \item Como estavam dispostas as comunidades indígenas e/ou não-indígenas locais antes da Revolução Verde no Brasil (anos 1970)?
  \item Como estão dispostas as comunidades indígenas e/ou não-indígenas locais atualmente?
  \item Como a relação área de produção x mata nativa foi sendo alterada, desde o período da Revolução Verde até os dias atuais?
  \item Qual o percentual de ocupação do solo por comunidades antes da Revolução Verde e no dias atuais? Essa transformação de ocupação foi gradativa ou algum evento a potencializou?
  \item Qual o percentual de ocupação do solo por mata nativa antes da Revolução Verde e no dias atuais? Essa transformação de ocupação foi gradativa ou algum evento a potencializou?
  \item Existe alguma barreira física que impede ou dificulta a expansão do agronegócio?
  \item As áreas de proteção nas margens dos rios estão sendo respeitadas?
  \item Existe alguma explicação espacial para os locais que comunidades indígenas (e pequenos produtores) ocupam hoje?
  \item É possivel estabelecer relações espaciais entre a chegada dos grandes agentes do agronegócio e o desmatamento?
  \item É possivel estabeler um padrão de ocupação do solo pelo agronegócio? E estabelecer previsões? Ele avança em direção à amazônia?
\end{enumerate}

A partir destes questionamentos, o próximo passo seria levantar as possíveis fontes de dados para essas perguntas e pensar quais as melhores formas de representar cada um dos dados obtidos (gráficos, tabelas, mapas, croquis, grafos...)

Dentre as fontes de dados, acredito que imagens de satélite históricas, mapas geomorfológicos e as informações de uso do solo de comunidades que serão obtidas através do mapeamento coletivo proposto na pesquisa.

Adicionalmente, dados do IBGE e do Ministério da Agricultura e Agropecuária expondo números do agronegócio no Maranhão para poder complementar a análise que será feita com base nas perguntas acima.

\subsection{Módulo 2 - Análise de distribuições espaciais em mapas de pontos}

Para aplicar os conteúdos deste módulo na pesquisa, podemos considerar como exemplo inicial, as seguintes questões (que envolvem uma situação em um tempo) apresentadas no módulo anterior:

\begin{itemize}
  \item Como estavam dispostas as comunidades indígenas e/ou não-indígenas locais antes da Revolução Verde no Brasil (anos 1970)?
  \item Como estão dispostas as comunidades indígenas e/ou não-indígenas locais atualmente?
\end{itemize}

Estes dois questionamentos poderiam ser respondidos visualmente através de mapas de pontos e, a partir destes mapas, podemos tentar identificar padrões, dispersões e densidades das ocupações. O intervalo temporal de cerca de quarenta anos entre os dois mapas também permite identificar alterações no padrão de ocupação humana. Se essa alteração existir, poderíamos questionar também se algum fator externo (político, econômico, etc) contribuiu para essa alteração.

Todavia, para analisar com mais propriedade, pode ser viável a combinação destes mapas com mapas geomorfológicos e/ou pedológicos da área, afim de buscar alguma explicação da paisagem (onde aqui a paisagem é considerada meramente como a forma da terra) para a ocupação (original e para a possivel alteração), por exemplo.

Esse tipo de análise é necessária pois há um pressuposto de que hoje as comunidades locais vivem em regiões mais acidentadas geomorfologicamente, pois teriam sido gradativamente expulsas das grandes áreas planas pelos agentes do agronegócio, visto que nessas áreas é possível produzir mais facilmente com as grandes máquinas. 

Se for possível obter dados para realizar este ensaio a cada década, por exemplo, talvez fosse válido criar um estudo baseado no vetor de mobilidade do centro geográfico ponderado de nuvens de pontos, analisando a posição do centro geográfico em cada década para acompanhar a evolução do fenômeno. Nesse estudo também seria possível identificar as dispersões espaciais ano a ano e a partir destas informações espaciais, buscar explicações em campo e na literatura para entender o fenômeno.

Finalizando, com base em respostas das duas questões iniciais, podemos expandir para buscar as respostas de outras questões que foram propostas inicialmente e de novos questionamentos que surgem a partir dos resultados obtidos no início.


\subsection{Módulo 3 - Séries espaciais e superfícies geográficas}

pode ser util

Dependendcia espacial e temporal entre eventos. 
A evolução do agronegócio E a eliminação do campesinato pode ser considerada uma evolucao temporal (p2)

Dependencia espacial -> para analisar o dematamento. Se aqui eh mato, agora nao eh, os vizinhos tendem a deixar de ser tb
autocorrelacao tempral serve para ajudar isso

buffer para delimitar áreas de protecao ambiental em torno de rios, pexemplo (isodistancias)

nao rola

autocorrelacao de mapas binarios
semivariogramas

\subsection{Módulo 4 - Funções básicas para modelagem de mapas em SIG}

pode ser util

acho que so as funcoes booleanas...\#mamilos

nao rola

acho que as coisas fuzzy nao serviriam, pq nao vou precisar de coisas ''aleatórias''. justificar bem isso. nao preciso mesclar classes com fuzzies

indice de agregacao de krmaer tb nao...nao tenho agregacoes assim acho...que precise reclassificar e talz

\subsection{Módulo 5 - Introdução à modelagem de interação espacial}

pode ser util

a noção de distancia em rota pode ser mto util para explicar as redes de exportacao, por exemplo
tb tem propabilidade de interacao espacial

alguma coisa para repsentar como matriz de conectividade binária em grafos

acho que consigo trazer um pouco das coisas humanas para análises

nao rola


Voltar no modulo 1, na sintese e explicar um pouco mais da teoria das perguntas espaciais...
