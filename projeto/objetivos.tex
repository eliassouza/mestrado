\section{Objetivos}

\subsection{Gerais}

\begin{itemize}
 \item Compreender, como parte do objeto de estudo da pesquisa, a dinâmica territorial da região e como a paisagem e a cultura local estão sendo alteradas conforme o agronegócio avança na região;
 \item Auxiliar, como parte do resultado da pesquisa, na propagação no ciberespaço\footnote{O ciberespaço é o novo meio de comunicação que surge da interconexão mundial dos computadores. O termo especifica não apenas a infra-estrutura material da comunicação digital, mas também o universo oceânico de informações que ela abriga, assim como os seres humanos que navegam e alimentam esse universo \cite[p. 17]{levy}.} de informações geradas pelos movimentos sociais regionais acerca da ação do agronegócio;
\end{itemize}

\subsection{Específicos}

\begin{itemize}
 \item Revisar a bibliografia sobre redes geográficas, a influência da técnica na composição do espaço, movimentos sociais em rede e impactos do agronegócio sobre um bioma;

 \item Mapear os pontos de avanço do agronegócio / áreas urbanas sobre a área de vegetação na região de estudo;

 \item Desenvolver uma ferramenta que tenha a função de um “portal” de informações sobre o impacto ambiental na região, com base em mapas gerados de forma colaborativa pela a sociedade e dados estatísticos sobre o desmatamento na região. Essa ferramenta poderá ser utilizada pela própria população, facilitando o acesso à técnica e à informação;

 \item Auxiliar no trabalho dos movimentos sociais (Sindicato de Trabalhores Rurais e COPEVIDA) de São Raimundo das Mangabeiras, utilizando o portal de informações para disponibilizar no ciberespaço material oriundo da sociedade civil organizada, sobre os problemas causados pelo agronegócio na região de estudo, levando o movimento social para a rede com objetivo de integrar e expandir o movimento;

 \item Atuar como projeto-piloto  para  aplicação da técnica e das possíveis conseqüências potenciais do uso da técnica pelos movimentos sociais como ferramenta capaz de difundir de suas vozes sobre os problemas locais.
\end{itemize}




