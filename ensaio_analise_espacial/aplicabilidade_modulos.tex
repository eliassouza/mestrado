\section{Aplicabilidade do conteúdo dos módulos da disciplina na pesquisa proposta}

\subsection{Módulo 1 - Conceitos e paradigmas da análise geoespacial na geografia}

Conforme relatei no capítulo anterior, acredito que as lições sobre planejamento de pesquisa seja o grande diferencial desse módulo. Como trabalho com a expansão do agronegócio e como a sociedade civil está se organizando para combater essa expansão, acredito que teria que planejar minha pesquisa em torno de alguns questionamentos básicos:

\begin{itemize}
  \item Como era a ocupação do solo na região de estudo antes da Revolução Verde no Brasil (anos 1970)?
  \item Como é a ocupação do solo na região de estudo no período atual?
  \item Como estavam dispostas as comunidades indígenas e/ou não-indígenas locais antes da Revolução Verde no Brasil (anos 1970)?
  \item Como estão dispostas as comunidades indígenas e/ou não-indígenas locais atualmente?
  \item Como a relação área de produção x mata nativa foi sendo alterada, desde o período da Revolução Verde até os dias atuais?
  \item Qual o percentual de ocupação do solo por comunidades antes da Revolução Verde e no dias atuais? Essa transformação de ocupação foi gradativa ou algum evento a potencializou?
  \item Qual o percentual de ocupação do solo por mata nativa antes da Revolução Verde e no dias atuais? Essa transformação de ocupação foi gradativa ou 
algum evento a potencializou?
  \item Existe alguma barreira física que impede ou dificulta a expansão do agronegócio?
  \item Existe alguma explicação espacial para os locais que comunidades locais (e pequenos produtores) e indígenas ocupam hoje?
  \item É possivel estabelecer relações espaciais entre a chegada dos grandes agentes do agronegócio e o desmatamento?
  \item É possivel estabeler um padrão de ocupação do solo pelo agronegócio? E estabelecer previsões? Ele avança em direção à amazônia?
\end{itemize}


A partir disso, identificar as possíveis fontes de dados (. . . . )

\subsection{Módulo 2 - Análise de distribuições espaciais em mapas de pontos}

\subsection{Módulo 3 - Séries espaciais e superfícies geográficas}

\subsection{Módulo 4 - Funções básicas para modelagem de mapas em SIG}

\subsection{Módulo 5 - Introdução à modelagem de interação espacial}



