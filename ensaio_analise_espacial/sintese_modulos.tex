\section{Síntese do conteúdo dos módulos da disciplina}

\subsection{Módulo 1 - Conceitos e paradigmas da análise geoespacial na geografia}

Para este módulo, eu esperava algum tipo de retomada das formas de análise geoespacial ao longo da história, o que, de fato, aconteceu, através de uma reflexão das escolas do pensamento geográfico que trabalharam com a temática da análise espacial e suas características, semelhanças e diferenças. 

Ao longo da aula, me chamaram a atenção as discussões sobre metodologia de pesquisa e a importância de diferenciar o instrumento (SIG) e a metodologia (formas de análise, etc). Descrever o instrumento é algo sem sentido e o importante é discutir o método de obtenção de dados e de análise. 

Enquanto discutíamos o raciocínio pré-pesquisa e fontes de dados, também conversamos sobre níveis de detalhe nos mapas e o quanto de ruído isso pode gerar. Aqui, me chamou a atenção a importância dada ao planejamento pré-pesquisa, desde o que pensamos em responder (as perguntas espaciais), até as formas como podemos representar as respostas. A mensagem que ficou para mim é que saber fazer as perguntas corretas é mais importante do que conhecer um SIG a fundo.

As perguntas espaciais podem questionar sobre qual é a situação de determinadas características em um momento no tempo (em em uma série de momentos), em um determinado local (para comparações, podemos questionar por vários locais). Também podem buscar por associações entre duas ou mais características ou locais, associando situações ou locais, assim como buscar identificar a correlação espacial entre duas ou mais características e como isso evolui ao longo do tempo.

\subsection{Módulo 2 - Análise de distribuições espaciais em mapas de pontos}

Aqui, eu tinha a expectativa de encontrar diversas aplicações de mapas de pontos. Em certa parte, foi o que ocorreu. O início do módulo trabalha com definições sobre o ponto, a substância de um ponto, arranjo de pontos e aplicações desses itens para, por exemplo, descobrirmos padrões de ocorrência. Neste ponto também me chamou atenção uma parte da explicação sobre a análise de mapas de pontos (onde a discussão era sobre padrão, dispersão e densidade de pontos) que dizia: ''o fenômeno espacial está agrupado em alguma região da área estudada, se está disperso em todos os locais, esse fenômeno não é espacial''.

Ao estudarmos as distribuições espaciais em mapas de pontos, pudemos pensar nos padrões aleatórios e agregados e na faixa de agregação (0 a 2,15), onde podemos ter um indicador de agregação de pontos (Rn)  indicando o quanto uma área de afasta do padrão esperado. Aqui também falamos como impedâncias físicas podem alterar a distribuição de pontos esperada e servir como explicação para distribuição obtida.

Depois foi apresentada a freqüência de pontos por quadrícula, onde discutimos que quando a variância dos valores na grid quadriculada é grande, temos alta agregação. Também falamos sobre centro médio e centro geográfico ponderado de nuvens de pontos, e que deveríamos utilizar o centro geográfico ponderado quando há substâncias, significados atribuídos aos pontos.

\subsection{Módulo 3 - Séries espaciais e superfícies geográficas}

O módulo 3 foi o mais complexo do curso, com muitas teorias matemáticas (que não são tão complexas conforme vamos estudando, mas que assustam um pouco à primeira vista). 

Começamos o módulo revendo conceitos estatísticos de dependência e independência entre eventos, para estudarmos dependência espacial e dependência temporal entre eventos. Aqui, vimos que eventos próximos temporalmente entre si demonstram maior influência de um sobre o outro, caracterizando uma dependência temporal entre eles. No caso de dependência espacial, podemos trabalhar com a idéia da autocorrelação espacial, onde quanto mais próximos espacialmente de um evento, mais parecidos são os dados no entorno deste local. Porém, precisamos sempre pensar: Até que ponto a posição interfere no fenômeno?

Mais adiante no conteúdo do módulo, vimos como trabalhar com estimativas de dados utilizando a autocorrelação espacial, semivariogramas, interpolações (que geram melhores resultados quando há dependência espacial).

Em um módulo mais complexo, a lição que fica é de que os números, apesar de importantes, não determinam o lugar, que é a variável mais importante da nossa análise.

\subsection{Módulo 4 - Funções básicas para modelagem de mapas em SIG}

Neste módulo, esperava alguns exemplos um pouco mais práticos, mas acredito que a abordagem utilizada foi mais proveitosa. Ao discutirmos as funções fuzzy e booleanas, pude pensar em mais opções de representação, conforme a característica do evento que está sendo mapeado.

Se o processo de decisão fuzzy parte do pressuposto que não há certeza em nenhuma variável dentro da área, ele atua de forma probabilística. A classificação fuzzy pode ser linear ou trapezoidal. Aqui, entendi que a fuzzificação linear faz mais sentido quando o objetivo é transformar séries grandes que possuem uma distribuição normal, com pouca dependência espacial. Do contrário, a representação trapezoidal gera melhores resultados. A utilização de buffers para determinar distâncias para uma classe também foi discutida.

Também conversamos sobre o índice de agregação local de Cramer, que avalia a agregação de características qualitativas e é usado quando o número de categorias de dois mapas são diferentes.

Por fim, funções booleanas, com operações lógicas e/ou condicionais (OR, AND, NOT) para a combinação de objetos ou de campos de diferentes mapas em um só.

\subsection{Módulo 5 - Introdução à modelagem de interação espacial}

No último módulo, começamos falando sobre a diferença entre as distâncias euclidianas e em rota, assunto que me interessa bastante pois trabalha com a idéia de redes, circuitos espaciais e outros conceitos em que facilmente conseguimos analisar com olhos da geografia humana.

A distância euclidiana é a medida do comprimento da reta, medida em metros, relacionada ao espaço absoluto, e não considera os obstáculos que podem existir na trajetória. Já a distância em rota é medida em tempo, relacionada ao espaço relativo, calculada com base no traçado das redes geográficas e dos circuitos espaciais. Essas redes e circuitos podem ser materiais ou imateriais, depende do dado sendo analisado.

Depois, o Princípio do Descaimento com a Distância, de Chapman, que trabalha com a idéia de que a interação espacial entre dois lugares diminui com o aumento da distância métrica ou horária entre eles. Mas com a ressalva que este princípio depende da duração da rota e do tipo de carga (o exemplo de aula dos tijolos e da televisão, que indicam que produtos de valor alto viajam mais longe esclarece essa ressalva). Ainda conversamos sobre as influências locais nas interações espaciais, como a especulação imobiliária. Aqui, partimos para o estudo das probabilidades de interação espacial, considerando motivadores de locomoção de um ponto até outro em uma cidade, por exemplo. Depois, vimos alguns modelos gravitacionais.

Por fim, o módulo me fez refletir se um mapa é sempre a melhor representação de um fenômeno espacial. Gráficos, matrizes, grafos podem substituir a representação tradicional em muitos casos.



