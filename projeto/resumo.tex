\begin{resumoumacoluna}
Em tempos de desenvolvimento do agronegócio e conflitos de terra, o detentor da técnica exerce maior poder, legitimando seus interesses no território. Se a sociedade está, ainda que de forma seletiva, cada vez mais conectada à rede, acreditamos ser de fundamental a possibilidade de os movimentos sociais também terem meios de se apropriarem dessas técnicas, para que possam trabalhar com a informação com um potencial parecido com o dos agentes hegemônicos.

Devido à expansão do agronegócio no país, observa-se cada vez mais a ocorrência de  conflitos entre representantes da sociedade civil organizada, a exemplo de diversos movimentos sociais do campo e da cidade, e grandes empreendimentos agropecuários e agroindustriais além dos agentes do Estado, com  grandes projetos de infraestrutura executados pelo aparelho estatal.


Busca-se nesse projeto, a partir de orientações teóricas de Milton Santos e Manuel Castells sobre a técnica e a sociedade em rede, ajudar a reduzir a distância no acesso à técnica (e consequentemente ao poder) que existe atualmente entre os dois grupos de conflito (agentes do agronegócio e movimentos sociais). Para tanto, propõe-se a elaboração de uma ferramenta computacional que possibilite a organização e divulgação de informações nas redes sociais, visando facilitar a organização e amplitude de trabalho dos movimentos sociais da região de São Raimundo das Mangabeiras, na região sul do Maranhão.

 \vspace{\onelineskip}
 \noindent
 \textbf{Palavras-chaves}: Técnica. Movimentos sociais. Sociedade em rede. Poder e contrapoder. Agronegócio.
\end{resumoumacoluna}

