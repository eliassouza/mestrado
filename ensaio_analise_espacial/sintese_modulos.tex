\section{Síntese do conteúdo dos módulos da disciplina}

\subsection{Módulo 1 - Conceitos e paradigmas da análise geoespacial na geografia}

Para este módulo, eu esperava algum tipo de retomada das formas de análise geoespacial ao longo da história, o que, de fato, aconteceu, através de uma reflexão das escolas do pensamento geográfico que trabalharam com a temática da análise espacial e suas características, semelhanças e diferenças. 

Ao longo da aula, me chamaram a atenção as discussões sobre metodologia de pesquisa e a importância de diferenciar o instrumento (SIG) e a metodologia (formas de análise, etc). Descrever o instrumento é algo sem sentido e o importante é discutir o método de obtenção de dados e de análise. 

Enquanto discutíamos o raciocínio pré-pesquisa e fontes de dados, também conversamos sobre níveis de detalhe nos mapas e o quanto de ruído isso pode gerar. Aqui, me chamou a atenção a importância dada ao planejamento pré-pesquisa, desde o que pensamos em responder (as perguntas espaciais), até as formas como podemos representar as respostas. Aqui, a mensagem que ficou para mim é que saber fazer as perguntas corretas é mais importante do que conhecer um SIG a fundo.

\subsection{Módulo 2 - Análise de distribuições espaciais em mapas de pontos}

Aqui, eu tinha a expectativa de encontrar diversas aplicações de mapas de pontos. Em certa parte, foi o que ocorreu. O início do módulo trabalha com definições sobre o ponto, a substância de um ponto, arranjo de pontos e aplicações desses itens para, por exemplo, descobrirmos padrões de ocorrência. Neste ponto também me chamou atenção uma parte da explicação sobre a análise de mapas de pontos (onde a discussão era sobre padrão, dispersão e densidade de pontos) que dizia: ''o fenômeno espacial está agrupado em alguma região da área estudada, se está disperso em todos os locais, esse fenômeno não é espacial''.

Ao estudarmos as distribuições espaciais em mapas de pontos, pudemos pensar nos padrões aleatórios e agregados e na faixa de agregação (0 a 2,15), onde podemos ter um indicador de agregação de pontos (Rn)  indicando o quanto uma área de afasta do padrão esperado. Aqui também falamos como impedâncias físicas podem alterar a distribuição de pontos esperada e servir como explicação para distribuição obtida.

Depois foi apresentada a freqüência de pontos por quadrícula, onde discutimos que quando a variância dos valores na grid quadriculada é grande, temos alta agregação. Também falamos sobre centro médio e centro geográfico ponderado de nuvens de pontos, e que deveríamos utilizar o centro geográfico ponderado quando há substâncias, significados atribuídos aos pontos.

\subsection{Módulo 3 - Séries espaciais e superfícies geográficas}



\subsection{Módulo 4 - Funções básicas para modelagem de mapas em SIG}

\subsection{Módulo 5 - Introdução à modelagem de interação espacial}
