\section{Proposta de plano de trabalho}

A elaboração da pesquisa aqui proposta pode ser subdividida em algumas seções e atividades a serem distribuídas ao longo de vinte e quatro meses de trabalho e contemplam levantamento de bibliografias e dados, trabalhos de campo, atividades de cunho prático e/ou técnico, além de análise de dados e redação de relatórios e artigos expondo a evolução do trabalho.

A seção de \textbf{levantamento bibliográfico e de dados} consiste em realizar um levantamento das produções bibliográficas, de teorias e métodos geográficos que:


\begin{itemize}
 \item Ressaltem as características da região estudada e as possibilidades e impedimentos dos movimentos sociais locais e discutam evidências de impactos ambientais causados pelo agronegócio;
 \item Abordem a técnica e sua influência na configuração do território, especialmente na região de estudo;
 \item Dissertem sobre as possibilidades de organização da sociedade em rede e o papel das redes sociais para a divulgação / organização de movimentos sociais;
\end{itemize}

Já a seção de \textbf{trabalhos técnicos} consiste na elaboração de um software com base em metodologias ágeis\footnote{São metodologias de desenvolvimento de software baseadas em processos iterativos e incrementais de planejamento, execução, validação e reflexões sobre as decisões tomadas. Essas metodologias, como o SCRUM e a programação extrema, visam o contínuo incremento de valor ao produto de forma iterativa, em curtos períodos de tempo (cada iteração pode levar entre duas ou três semanas), afim de maximizar a qualidade e a real utilidade do software. Com o uso dessas metodologias, não é necessário esperar a finalização do desenvolvimento do software como um todo para iniciar sua utilização.} de desenvolvimento de software, computação em nuvem\footnote{De forma simplória, a computação em nuvem é baseada na utilização de recursos de infra-estrutura e/ou serviços disponíveis na própria internet, eliminando a necessidade de criar uma estrutura física de servidores para publicar uma aplicação na internet.} e tecnologias livres\footnote{Tecnologias onde existe liberdade para utilizar, estudar, modificar e redistribuir a tecnologia original.}, sendo disponibilizado em ambiente de código-aberto\footnote{O código-fonte gerado para a construção do software estará disponível ao público na internet.}, para livre utilização pela comunidade.

Ainda nesta seção, propomos a confecção de croquis e mapas sobre a evolução dos impactos sócio-ambientais na área de estudo, baseado nas informações obtidas através do software elaborado e nos trabalhos de campo realizados. Já a \textbf{análise dos dados} ocorrerá de forma quase constante visando o direcionamento e a elaboração do trabalho da pesquisa. 

\subsection{Cronograma}

O cronograma prevê dois anos de atividades, dispostas temporalmente conforme os quadros ~\ref{qd:cronoI} e ~\ref{qd:cronoII}.

\begin{quadro}[!htbp]
\caption[CronogramaI]{Cronograma de atividades - Ano I}

\begin{tabular}{|l |c|c|c |c|c|c |c|c|c |c|c|c|}

\hline
\textbf{Atividade} &  \multicolumn{3}{|c|}{\textbf{1º Trim.}} & \multicolumn{3}{|c|}{\textbf{2º Trim.}} & \multicolumn{3}{|c|}{\textbf{3º Trim.}} & \multicolumn{3}{|c|}{\textbf{4º Trim.}} \\

\hline
{}  &  \textbf{1} &  \textbf{2}  &  \textbf{3}  &  \textbf{4}  &  \textbf{5}  &  \textbf{6}  &  \textbf{7}  &  \textbf{8}  &  \textbf{9}  &  \textbf{10}  &  \textbf{11}  &  \textbf{12}  \\

\hline
Levantamento de dados  & \cellcolor[gray]{0.6} & \cellcolor[gray]{0.6} & \cellcolor[gray]{0.6} & \cellcolor[gray]{0.6}  & \cellcolor[gray]{0.6} & \cellcolor[gray]{0.6} & \cellcolor[gray]{0.6} & \cellcolor[gray]{0.6} & \cellcolor[gray]{0.6} & {} & {} & {} \\

\hline
Pesquisa bibliográfica  &  \cellcolor[gray]{0.6} & \cellcolor[gray]{0.6} & \cellcolor[gray]{0.6} & \cellcolor[gray]{0.6}  & \cellcolor[gray]{0.6} & \cellcolor[gray]{0.6} & \cellcolor[gray]{0.6} & \cellcolor[gray]{0.6} & \cellcolor[gray]{0.6} & \cellcolor[gray]{0.6} & {} & {} \\

\hline 
Trabalhos de campo  &  {} & {} & {} & \cellcolor[gray]{0.6}  & \cellcolor[gray]{0.6} & {} & {} & {} & {} & {} &  \cellcolor[gray]{0.6} &  \cellcolor[gray]{0.6} \\

\hline
Trabalhos técnicos (softwares, mapas) & \cellcolor[gray]{0.6} & \cellcolor[gray]{0.6} & \cellcolor[gray]{0.6} & \cellcolor[gray]{0.6}  & \cellcolor[gray]{0.6} & \cellcolor[gray]{0.6} & \cellcolor[gray]{0.6} & \cellcolor[gray]{0.6} & \cellcolor[gray]{0.6} & {} & {} & {} \\

\hline
Análise de dados  & {} & {} & {} & {} & {} & \cellcolor[gray]{0.6} & \cellcolor[gray]{0.6} & \cellcolor[gray]{0.6} & \cellcolor[gray]{0.6} & \cellcolor[gray]{0.6} & \cellcolor[gray]{0.6} & \cellcolor[gray]{0.6}\\

\hline
Elaboração de artigos científicos & {} & {} & {} & {} & {} & {} & {} & {} & {} & {} & {} & \cellcolor[gray]{0.6}\\

\hline
Exame de qualificação & {} & {} & {} & {} & {} & {} & {} & {} & {} & {} & {} & {}\\

\hline
Elaboração e defesa da dissertação final & {} & {} & {} & {} & {} & {} & {} & {} & {} & {} & {} & {}\\

\hline
\end{tabular}
\label{qd:cronoI}
\end{quadro}


\begin{quadro}[!htbp]
\caption[CronogramaII]{Cronograma de atividades - Ano II}

\begin{tabular}{|l |c|c|c |c|c|c |c|c|c |c|c|c|}

\hline
\textbf{Atividade} &  \multicolumn{3}{|c|}{\textbf{1º Trim.}} & \multicolumn{3}{|c|}{\textbf{2º Trim.}} & \multicolumn{3}{|c|}{\textbf{3º Trim.}} & \multicolumn{3}{|c|}{\textbf{4º Trim.}} \\

\hline
{}  &  \textbf{1} &  \textbf{2}  &  \textbf{3}  &  \textbf{4}  &  \textbf{5}  &  \textbf{6}  &  \textbf{7}  &  \textbf{8}  &  \textbf{9}  &  \textbf{10}  &  \textbf{11}  &  \textbf{12}  \\

\hline
Levantamento de dados  & \cellcolor[gray]{0.6} & \cellcolor[gray]{0.6} & \cellcolor[gray]{0.6} & \cellcolor[gray]{0.6}  & \cellcolor[gray]{0.6} & \cellcolor[gray]{0.6} & \cellcolor[gray]{0.6} & \cellcolor[gray]{0.6} & \cellcolor[gray]{0.6} & {} & {} & {} \\

\hline
Pesquisa bibliográfica  &  \cellcolor[gray]{0.6} & \cellcolor[gray]{0.6} & \cellcolor[gray]{0.6} & \cellcolor[gray]{0.6}  & \cellcolor[gray]{0.6} & \cellcolor[gray]{0.6} & \cellcolor[gray]{0.6} & \cellcolor[gray]{0.6} & \cellcolor[gray]{0.6} & \cellcolor[gray]{0.6} & {} & {} \\

\hline 
Trabalhos de campo  &  {} & {} & {} & {} & {} & \cellcolor[gray]{0.6} & \cellcolor[gray]{0.6} & {} & {} & {} &  {} & {}  \\

\hline
Trabalhos técnicos (softwares, mapas) & \cellcolor[gray]{0.6} & \cellcolor[gray]{0.6} & \cellcolor[gray]{0.6} & \cellcolor[gray]{0.6}  & \cellcolor[gray]{0.6} & \cellcolor[gray]{0.6} & \cellcolor[gray]{0.6} & \cellcolor[gray]{0.6} & \cellcolor[gray]{0.6} & \cellcolor[gray]{0.6} & {} & {} \\

\hline
Análise de dados  & \cellcolor[gray]{0.6} & \cellcolor[gray]{0.6} & \cellcolor[gray]{0.6} & \cellcolor[gray]{0.6} & \cellcolor[gray]{0.6} & \cellcolor[gray]{0.6} & \cellcolor[gray]{0.6} & \cellcolor[gray]{0.6} &  \cellcolor[gray]{0.6} & {} & {} & {}\\

\hline
Elaboração de artigos científicos & \cellcolor[gray]{0.6} & {} & {} & {} & {} & {} & {} & {} & \cellcolor[gray]{0.6} & {} & {} & {}\\

\hline
Exame de qualificação & {} & {} & {} & {} & {} & \cellcolor[gray]{0.6} & \cellcolor[gray]{0.6} & {} & {} & {} & {} & {}\\

\hline
Elaboração e defesa da dissertação final & {} & {} & {} & {} & {} & {} & {} & \cellcolor[gray]{0.6} & \cellcolor[gray]{0.6} & \cellcolor[gray]{0.6} & \cellcolor[gray]{0.6} & \cellcolor[gray]{0.6}\\

\hline
\end{tabular}
\label{qd:cronoII}
\end{quadro}


