\section{Perguntas espaciais de referência}

Dentre as perguntas espaciais apresentadas na aula, acredito que as perguntas destacadas abaixo são importantes no contexto da minha pesquisa:

\begin{itemize}
\item \textbf{Tipo I} - Qual a área total de fragmentos florestais (C1), culturas de soja (C2) e silvicultura (C3) no município L em 2013 (T1)?
\item \textbf{Tipo III} - Como evoluíram, década a década, as áreas de cana de açúcar (C1), cerrado (C2), matas (C3) e pastagem (C4), no município (L1), entre 1960 (T1) e 2010 (T50)?
\item \textbf{Tipo IV} - Como evoluiu, década a década, a distribuição espacial das áreas de cerrado (C1) na região sudeste de Goiás (L1, L2... LN) entre 1960 (T1) e 2010 (T50)?
\item \textbf{Tipo V} - Qual é o grau de correspondência espacial da ocorrência de voçorocas (C1) e latossolos areno-argilosos (C2) no ano de 2012 (T1)?
\end{itemize}

Na análise da aplicabilidade do módulo I em minha pesquisa, apresentei diversas questões que acredito que deva responder ao longo do trabalho. São questões que possuem uma estrutura semelhante às apresentadas em aula, com análise de características em um local ao longo de uma data (ou em um período). Acredito que os sete tipos de questões aparecem durante a pesquisa, mas estes quatro grupos escolhidos serão os mais comuns.

Assim, retomo aqui duas das questões apresentadas anteriormente para alguns comentários:

\begin{itemize}
  \item Como estão dispostas as comunidades indígenas e/ou não-indígenas locais atualmente?
  \item Como a relação área de produção x mata nativa foi sendo alterada, desde o período da Revolução Verde até os dias atuais?
\end{itemize}

A primeira pergunta é do \textbf{Tipo I}, pois visa analisar a distribuição de duas características (comunidades indígenas e comunidades não indígenas) em uma data (atualmente), sem considerar períodos históricos. Como uma foto do momento.

Já a segunda pergunta trabalha com uma estrutura semelhante à do \textbf{Tipo III}, já que analisa a evolução de duas características (área do produção e mata nativa) em um interfalo de tempo (início da Revolução Verde até os dias atuais).
