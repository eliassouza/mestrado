\chapter{Metodologia}

\citeauthoronline{isnard} (\citeyear{isnard}) afirma que o espaço geográfico é gerado pela sociedade, e sua produção e organização ao longo do tempo um campo de conflitos e embates. O espaço é, desse modo, “um amálgama de elementos que se movem, interagem e são solidários e contraditórios, por que criam espaços diferenciados, cada qual com sua função, com sua relação social” \cite{mondardo}

\citeauthoronline{santos1996} (\citeyear{santos1996}) definiu o espaço como um “conjunto indissociável de sistemas de objetos e sistemas de ações”, e a técnica, uma categoria analítica da associação entre esses sistemas, se mostra um elemento fundamental na explicação do espaço. Ainda assim, o autor frisa que “a técnica é um elemento importante de explicação da sociedade e dos lugares, mas, sozinha, a técnica não explica nada” \cite[p. 27]{santos1996}. Ela precisa ser estudada em conjunto com outros elementos do espaço, incluindo o tempo.

Citando Pierre George (1974), Milton lembra que "a influência da técnica sobre o espaço se exerce de duas maneiras e em duas escalas diferentes: a ocupação do solo pelas infra-estruturas das técnicas modernas [...] e as transformações generalizadas impostas pelo uso e execução dos novos métodos de produção e de existência" \cite[p. 19]{santos1996}. Dessa forma, considera as consequências diretas do uso da técnica, como a própria instalação da infra-estrutura, e consequências indiretas, as possibilidades que surgem com a implantação de um sistema técnico. 

Ainda segundo \citeauthoronline{santos1996}(1996, p. 83), a técnica atua na produção do espaço modificando-o em termos de forma, função e paisagem, fatores determinantes de novas relações entre a sociedade e o espaço e entre a sociedade e si mesma. E como ressalta \citeauthoronline{marques} (2009, p. 21), “cada lugar revela uma técnica ou um conjunto de técnicas que o caracteriza particularmente e que contribui na formação de uma identidade própria [...] desta forma, a técnica constitui um dos elementos de explicação da sociedade e de cada um dos lugares”.

Hoje a informação é um elemento chave na composição da sociedade mundial e o meio técnico-científico-informacional é a “cara espacial da globalização” \cite{santos1996}, assim como a informação e os sistemas de comunicação adquirem importância fundamental na organização do espaço.

Ao estudar as teorias de Milton Santos, \citeauthor{maia} (2012, p.9) sintetiza a história evolutiva do conceito "meio-técnico-científico-informacional", que “inicia-se na década de 1970; é caracterizado pela aplicação da ciência à técnica, por isto meio técnico científico; mas este meio, estas técnicas são impregnadas de informação e transmitem, acumulam informação, por isto meio-técnico-científico-informacional”. Milton define a relação entre os espaços, os agentes hegemônicos e o meio-técnico-científico-informacional:

\begin{citacao}
Os espaços assim requalificados atendem sobretudo aos interesses dos atores hegemônicos da economia, da cultura e da política e são incorporados plenamente às novas correntes mundiais. \cite[p. 191]{santos1996}
\end{citacao}

\citeauthor{maia} (2012, p.5) ressalta ainda que a modificação acelerada do território, aliada à chegada e dispersão das técnicas de comunicação e informação, dá ao período atual uma forma diferenciada, que Milton Santos chama, em seu livro "A Natureza do Espaço", de instantaniedade dos momentos e dos lugares, universalidade e unicidade das técnicas.

No documentário Milton Santos: por uma outra globalização, Milton afirma que a técnica central e dominante nos dias atuais é a técnica da informação, e que o homem deixou de ser o centro do mundo, papel que hoje é exercido pelo dinheiro, em uma geopolítica proposta por economistas e defendida pela mídia (detentora da técnica da informação). Dessa forma, as grandes empresas utilizam a mídia e, consequentemente a técnica, para realizar um tipo de dominação sobre o território, almejando se perpetuar como agente hegemônico.

Manuel Castells, no livro A Sociedade Em Rede (1999, p.50-51) considera a informação como um modo de desenvolvimento, moldado pela reestruturação do modo capitalista de produção, no final do século XX, que resulta no surgimento de uma nova estrutura social. Segundo o autor, a concepção teórica que fundamenta essa abordagem pressupõe que as sociedades são organizadas em processos estruturados por relações historicamente determinadas entre três elementos:

\begin{enumerate}

\item \textbf{Produção}, baseado na ação do homem sobre a natureza para apropriar-se dela e transforma-la seu beneficio;
\item \textbf{Experiência}, a ação dos humanos sobre si mesmos, determinada pela interação entre suas identidades biológicas e culturais em relação a seus ambientes sociais e naturais;
\item \textbf{Poder}, relação entre humanos que, com base na produção e na experiência, impõe a vontade de alguns sobre os outros, pelo emprego potencial ou real de violência física ou simbólica.

\end{enumerate}

Para o autor, a “comunicação simbólica entre os seres humanos e o relacionamento entre esses e a natureza,com base na produção (e seu complemento, o consumo), experiência e poder, cristalizam-se ao longo da história em territórios específicos, e assim geram culturas e identidades coletivas” \cite[p. 52-53]{castells1999}. Ele ainda ressalta como a difusão da tecnologia amplifica o poder de uma sociedade de forma infinita, a medida que os usuários apropriam-se da tecnologia e a redefinem: “pela primeira vez na história, a mente humana é uma força direta de produção, não apenas um elemento decisivo no sistema produtivo” \cite{castells1999}

A integração do território, motivada por interesses geopolíticos e pela necessidade de circulação de bens, pessoas e informação, deu-se através da implantação e extensão de redes geográficas, definidas por \citeauthoronline{correa} (1999) como um conjunto de localizações sobre a superfície terrestre articulado por vias e fluxos. Mais do que isso, a rede é um produto e também uma condição social, historicamente construída, dotada de intencionalidade e regulada politicamente \cite{santos1996}. \citeauthoronline{dias} (1995, p.150) frisa que a formação de redes no território é acompanhada de seletividade espacial, já que as redes não ligam todos os pontos. Elas exercem o papel de conexão de alguns pontos e de exclusão de outros, tornando mais estratégica a localização geográfica.

As redes constituem a morfologia social da sociedade atual e modificam de forma considerável a operação e os resultados dos processos produtivos e de experiência, poder e cultura. Elas são “estruturas abertas capazes de expandir de forma ilimitada, integrando novos nós desde que consigam comunicar-se dentro da rede” \cite[p. 566]{castells1999}. Essa última afirmação de Castells exemplifica o caráter excludente de uma rede, onde só é aceito o que se encaixa em seu padrão ou seus interesses. As redes, conforme Milton \citeauthoronline{santos1996} (1996), são agentes de inclusão e também de exclusão.

Para \citeauthoronline{castells1999} (1999, p. 70), as grandes áreas do mundo e consideráveis segmentos da população que estão desconectados do novo sistema tecnológico são regiões culturais e espacialmente descontínuas, enquanto grupos sociais e territórios dominantes estão constantemente conectados. Para o autor (1999, p. 476), sob a lógica do novo sistema o que importa não é a localização real dos centros de produção, mas a versatilidade de suas redes.

As sociedades são constituídas a partir das relações de poder entre seus membros, uma vez que os que detêm o poder constroem as instituições segundo seus valores e interesses. \citeauthoronline{castells2013} (2013, p.13) afirma que o “poder é exercido por meio da coerção e/ou pela construção de significado na mente das pessoas, mediante mecanismos de manipulação simbólica”.  Aqui, é importante citar \citeauthoronline{raffestin1993} (1993, p. 212-213), para quem os nós não são meros pontos de conexão entre redes, mas também de poder.

Todavia, \citeauthoronline{castells2013} ressalta que as sociedades são contráditórias e conflitivas, de forma que onde há \textbf{poder}, há também o chamado \textbf{contrapoder}

\begin{citacao}
a capacidade de os atores sociais desafiarem o poder embutido nas instituições da sociedade com o objetivo de reivindicar a representação de seus próprios valores e interesses [...] A verdadeira configuração do Estado e de outras instituições que regulam a vida das pessoas depende dessa constante interação de poder e contrapoder \cite[p .13]{castells2013}.
\end{citacao}

Nos últimos anos, temos vivenciado um novo poder na geopolítica mundial. São os movimentos sociais organizados através das redes sociais na internet. Espaços onde há autonomia,

\begin{citacao}
muito além do controle de governos e empresas, que, ao longo da história, haviam monopolizado os canais de comunicação como alicerces de seu poder. Compartilhando dores e esperanças no livre espaço público da internet, conectando-se entre si e concebendo projetos a partir de múltiplas fontes do ser, indivíduos formaram redes, a despeito de suas opiniões pessoais ou filiações organizacionais \cite[p .10]{castells2013}
\end{citacao}

Nesse novo modelo de organização, as comunicações de massa são interativas e baseadas em redes horizontais de poder, mais difíceis de serem controladas por governos e/ou empresas.

Para \citeauthoronline{castells2013} (2013, p.17-18), enquanto o poder é exercido programando e alterando redes, o contrapoder, uma tentativa de alterar as relações de poder predominantes, é realizado através da reprogramação das redes em torno de outros interesses e valor, que não os dos agentes hegemônicos. O contrapoder é exercido também pelos movimentos sociais, por meio de um processo de comunicação livre do controle dos que detêm o poder institucional, de forma que os movimentos possam ser construídos e agir na sociedade com menor influência do agente detentor de poder. 

Na sociedade em rede

\begin{citacao}
a autonomia de comunicação é basicamente construída nas redes da internet e nas platafomas de comunicação sem fio [...]  As redes sociais digitais oferecem a possibilidade de deliberar sobre e coordenar as ações de forma amplamente desimpedida. Entretanto, esse é apenas um componente do processo comunicativo pelo qual os movimentos sociais se relacionam com a sociedade em geral. Eles também precisam construir um espaço público, criando comunidades livres no espaço urbano. Uma vez que o espaço público institucional, o espaço constitucionalmente designado para a deliberação, está ocupado pelos interesses das elites dominantes e suas redes, os movimentos sociais precisam abrir um novo espaço público que não se limite à internet, mas se torne visível nos lugares da vida social \cite[p .18-19]{castells2013}.
\end{citacao}

A organização de movimentos sociais através das redes sociais permite uma ação e contatos mais ágeis e amplos do que a forma tradicional de organização (através de panfletos, boatos, reuniões). Porém, um dos grandes desafios é integrar e coordenar o trabalho tanto no ciberespaço quanto no espaço real, físico.



