\section{Referências adicionais}

\begin{enumerate}

\item CORNELISSEN, A.M.G; VAN DEN BERG, J.; KOOPS,W.J.; GROSSMAN, M., UDO, H.M.J. \textbf{Assessment of the contribution of sustainability indicators to sustainable development: a novel approach using fuzzy set theory, Agriculture, Ecosystems \& Environment}, v. 86, n. 2, 2001, p. 173-185. Disponível em: <http://www.sciencedirect.com/science/article/pii/S0167880900002723>. Acesso em: 16 jun. 2014.

\item GAO , S. GOODCHILD, M. \textbf{Asking Spatial Questions to Identify GIS Functionality}. 2013. Disponível em: <http://www.researchgate.net/publication/ AskingSpatialQuestionstoIdentifyGISFunctionality.pdf> Acesso em: 16 jun. 2014.

\item ZADEH, L. A. \textbf{Fuzzy sets. Informat Control}, p.338-353, 1965. Disponível em: <http://www.sciencedirect.com/science/article/pii/S001999586590241X>. Acesso em: 16 jun. 2014.

\end{enumerate}

A bibliografia número 1 se encaixa no módulo 4 do curso, pois trabalha com a temática da distribuição fuzzy. Ele trabalha com a teoria fuzzy para criar modelos matemáticos que sejam úteis no estudo da agricultura e projetos ecologicamente sustentáveis. Aparentemente os modelos matemáticos criados neste projeto ainda não estão em uso.

A bibliografia número 2 se encaixa nos módulos 1 e 5. Fala sobre a dificuldade de transportar as perguntas espaciais para o SIG. O artigo, bastante técnico, propõe um framework computacional que traga uma interface visual para o SIG, onde o usuário monta sua questão que é automaticamente transformada em uma linguagem que o SIG entenda.

A bibliografia número 3 também se encaixa no módulo 4 do curso, pois trabalha com a temática da distribuição fuzzy. Um texto bastante matemático que visa explicar o funcionamento da lógica de distribuição fuzzy, ao descrever um conjunto fuzzy como uma classe de objetos caracterizados por uma função característica que atribui um valor gradual (entre 0 e 1) para cada um desses objetos e que é a partir destes valores que as demais operações (uniões, etc) são realizadas.
